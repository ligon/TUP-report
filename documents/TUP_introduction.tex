\documentclass[]{article}
\usepackage{lmodern}
\usepackage{amssymb,amsmath}
\usepackage{ifxetex,ifluatex}
\usepackage{fixltx2e} % provides \textsubscript
\ifnum 0\ifxetex 1\fi\ifluatex 1\fi=0 % if pdftex
  \usepackage[T1]{fontenc}
  \usepackage[utf8]{inputenc}
\else % if luatex or xelatex
  \ifxetex
    \usepackage{mathspec}
  \else
    \usepackage{fontspec}
  \fi
  \defaultfontfeatures{Ligatures=TeX,Scale=MatchLowercase}
\fi
% use upquote if available, for straight quotes in verbatim environments
\IfFileExists{upquote.sty}{\usepackage{upquote}}{}
% use microtype if available
\IfFileExists{microtype.sty}{%
\usepackage{microtype}
\UseMicrotypeSet[protrusion]{basicmath} % disable protrusion for tt fonts
}{}
\usepackage[margin=1in]{geometry}
\usepackage{hyperref}
\hypersetup{unicode=true,
            pdftitle={TUP\_report},
            pdfauthor={Reajul Chowdhury, Elliott Collins, Ethan A. Ligon, Munshi Sulaiman},
            pdfborder={0 0 0},
            breaklinks=true}
\urlstyle{same}  % don't use monospace font for urls
\usepackage{graphicx,grffile}
\makeatletter
\def\maxwidth{\ifdim\Gin@nat@width>\linewidth\linewidth\else\Gin@nat@width\fi}
\def\maxheight{\ifdim\Gin@nat@height>\textheight\textheight\else\Gin@nat@height\fi}
\makeatother
% Scale images if necessary, so that they will not overflow the page
% margins by default, and it is still possible to overwrite the defaults
% using explicit options in \includegraphics[width, height, ...]{}
\setkeys{Gin}{width=\maxwidth,height=\maxheight,keepaspectratio}
\IfFileExists{parskip.sty}{%
\usepackage{parskip}
}{% else
\setlength{\parindent}{0pt}
\setlength{\parskip}{6pt plus 2pt minus 1pt}
}
\setlength{\emergencystretch}{3em}  % prevent overfull lines
\providecommand{\tightlist}{%
  \setlength{\itemsep}{0pt}\setlength{\parskip}{0pt}}
\setcounter{secnumdepth}{0}
% Redefines (sub)paragraphs to behave more like sections
\ifx\paragraph\undefined\else
\let\oldparagraph\paragraph
\renewcommand{\paragraph}[1]{\oldparagraph{#1}\mbox{}}
\fi
\ifx\subparagraph\undefined\else
\let\oldsubparagraph\subparagraph
\renewcommand{\subparagraph}[1]{\oldsubparagraph{#1}\mbox{}}
\fi

%%% Use protect on footnotes to avoid problems with footnotes in titles
\let\rmarkdownfootnote\footnote%
\def\footnote{\protect\rmarkdownfootnote}

%%% Change title format to be more compact
\usepackage{titling}

% Create subtitle command for use in maketitle
\newcommand{\subtitle}[1]{
  \posttitle{
    \begin{center}\large#1\end{center}
    }
}

\setlength{\droptitle}{-2em}

  \title{TUP\_report}
    \pretitle{\vspace{\droptitle}\centering\huge}
  \posttitle{\par}
    \author{Reajul Chowdhury, Elliott Collins, Ethan A. Ligon, Munshi Sulaiman}
    \preauthor{\centering\large\emph}
  \postauthor{\par}
      \predate{\centering\large\emph}
  \postdate{\par}
    \date{December 03, 2018}


\begin{document}
\maketitle

\section{Introduction}\label{introduction}

Despite remarkable progress on reducing poverty over the last decades,
the United Nations (UN) and the World Bank admit that we are far behind
and might fail in achieving the goal of ending extreme poverty\footnote{population
  living on less than US\$1.90 a day} by 2030 (World Bank 2018). The
reasons behind this slow progress are mainly two. First, it is difficult
to reach the poor - specially the poorest among the poor. Second,
development models or interventions which effectively alleviate poverty
and provide sustainable livelihoods to extreme poor are rare in number.
So, the biggest challenge in poverty alleviation is reaching the
`poorest' with an `effective' intervention. The poorest or the
`ultra-poor'\footnote{The term ultra-poor was coined by Lipton (1983) in
  reference to a group of people that spend 80 per cent of their total
  income for food and eat below 80 per cent of their caloric needs.
  Later, a report by the International Food Policy Research Institute,
  defined as ultra-poor those living with less than US\$0.50 a day
  (Akhter U. Ahmed et al. 2007)} people are the most vulnerable subgroup
within the population living under extreme poverty line. They are
severely food insecure, own little or no productive assets, have no
education and skills, and earn money from low return activities like
small-scale cultivation or causal day labor.

NGOs and development organizations tried to help the ultra-poor
households with traditional microcredit product - which is a proven tool
in reducing vulnerability, creating economic opportunities, and
increasing income of poor people by giving them access to finance
(Hashemi and Umaira 2011). But the traditional microcredit does not fit
the needs of the poorest, and has failed to reach the ultra-poor
(Hashemi and Umaira 2011). Microcredit products are mainly small,
high-interest loans, usually designed for micro entrepreneurs- who are
already involved with or more likely to start some self-employment
activities, and will be able to make use of the credit to take advantage
of market opportunities (Banerjee et al. 2015; Hashemi and Umaira 2011).
The ultra-poor households are much poorer than the targeted groups of
microcredit providers and therefore, have been excluded from the most
microcredit programs (Evans et al. 1999; Thorp, Stewart, and Heyer 2005;
Kaur 2016). Even when offered with microcredit, the ultra-poor
households lack the human capital and entrepreneurial skills to
effectively use the credit, and might fall in credit-based poverty trap.
Another strategy through which governments and donor organizations reach
the ultra-poor is safety net programs such as food aid, school feeding,
temporary employment programs, free health services, unemployment
benefits, pensions, etc. Countries hosting most of the world's poor
population cannot often afford such safety net programs in long run. The
safety net programs often fail to reach the ultra-poor\footnote{Evidence
  shows that India's Integrated Rural Development Program (IRDP) had
  poor performance in targeting the poorest.(Dreze 1990; {\textbf{???}})}.
Such programs are also widely criticized for focusing only on protecting
the poor rather than helping them to improve their livelihoods and get
out of extreme poverty.

Over the last few decades, NGOs and development organizations learned
that a single product or service cannot fit the dynamic needs of the
ultra-poor and help them overcoming extreme poverty. Giving them a
sustainable livelihoods will require a comprehensive, integrated set of
interventions. Building on this insight, BRAC -- a Bangladesh origin
NGO, pioneered a program called Targeting the Ultra-poor (TUP) to build
secure, sustainable, and resilient livelihoods for the ultra-poor. The
TUP program follows a `Graduation' approach that combines multifaceted
support services addressing both immediate needs of the ultra-poor by
giving them consumption supports, and their long-term need for a
sustainable livelihoods by providing technical skills training, and a
grant of productive assets. BRAC first implemented the TUP program in
2002 in Bangladesh. Several non-experimental studies (Akhter U Ahmed et
al. 2009; Mallick 2013; Matin and Hulme 2003; Rabbani, Sulaiman, and
Prakash 2006) found the TUP program very effective in increasing
household consumption, assets holdings, and self-employment among the
ultra-poor. The holistic treatment of poverty in the Graduation approach
(or BRAC's TUP program) drew the attention of the donor community, and
other stakeholders in the extreme poverty affected countries. The TUP
model has later been replicated and adapted by 114 programs in 45
countries by NGOs, governments, and donor organizations\footnote{BRAC
  website: \url{http://www.brac.net/program/targeting-ultra-poor/}} .
Through its TUP program in Bangladesh, BRAC alone reached a total of
1.77 million ultra-poor households till 2016, and aims to reach 0.5
million more households by 2020\footnote{\url{http://www.brac.net/images/index/tup/brac_TUP-briefNote-Jun17.pdf}}.

Taking advantage of the large scale replication of the TUP program in
Bangladesh and other countries, a number of experimental studies (
(Banarjee, Duflo, Chattopadhyay, \& Shapiro, 2011) (Banerjee et al.
2011; Bandiera et al. 2013; Bandiera et al. 2017; Banerjee et al. 2015)
have been conducted to assess the impact of the program. In Bangladesh,
Bandiera et al. (2013) found that after four years of the program
inception, the beneficiary households expanded their self-employment
activities, increased labor supply, accumulated more productive assets,
which led to increased household income and per capita consumption. A
follow-up survey on the same households seven years after the program
began, found that the long-term effect of the program is at least as
large as the four-years effect (Bandiera et al. 2017). Banarjee et al.
(2015) documented the findings from 6 randomized trail studies assessing
the impact of the TUP program implemented in 6 countries\footnote{Ethiopia,
  Ghana, Honduras, India, Pakistan, and Peru}. The study found effect of
the program on income, assets holdings, food security, and consumption
similar to the Bangladesh study.

While the graduation approach (BRAC's TUP model) was getting popular and
being replicated in many countries, Haushofer \& Shapiro (2013), and
Chrsi Blattman et al. (2015) were experimenting the effect of a very
simple unconditional cash transfer (UCT) to help the ultra-poor
households fighting extreme poverty in Kenya and Uganda. The rational
for such simple cash transfer is straight forward -- putting the
development and social protection resources directly in the hands of
poor people and let them fight the dimensions of poverty in their own
way. Findings from both studies show significant effect of the cash
transfer on assets holdings (both productive and overall household
assets), consumption, and food security. For instance, the study from
Kenya found that compared to the control group mean, the recipients'
total assets holding increased by 58\%, and household consumption
increased by 20\% (Haushofer \& Shapiro,2013). A three-years follow up
survey on these households show a similar magnitude of effects persist
on the same welfare outcomes (Haushofer \& Shapiro, 2018). The study in
Uganda also found the unconditional cash transfer to be effective in
terms of increased self-employment activities, assets holdings, and
consumption of the recipients (Blattman et al. 2016). Though whether
such unconditional cash transfer can offer sustainable routs out of
poverty on a large scale still remains a question, the low operational
costs and significant effect on particular welfare outcomes (at least in
short run) made such transfer program an attractive alternative to
relatively more expensive Graduation approach/BRAC's TUP model. These
two competitive approaches of transfer program put an obvious question
to the development economists -- is it worth to invest on the additional
features of the TUP framework and which type of transfer is more
effective in a given context.

In this paper, we present results from a randomized trial study
comparing the BRAC's TUP program with unconditional cash transfers in
South Sudan. We aim to answer the research questions -- \textbf{does the
type of transfer matter and do the different types of transfer result
different welfare change?} Our results contribute to the general
literature in two important ways. First, to the best of our knowledge,
this is the first experiment attempting to directly compare these two
competitive approaches of transfer programs in the same setup. In our
study, a randomly selected group of households received cash transfers
equal in market value to the assets provided to the TUP households.
Second, South Sudan's long-history of ongoing conflict and volatile
economy, which may affect the value of the program for households in
important ways. South Sudan earned its independence after five decades
of civil war with North Sudan. The youngest nation of the world had some
of the worst development indicators in the world; under five and
maternal mortality in the county is world's highest (WHO 2014), the
whole country has only 124 miles of paved road, the unemployment rate in
the country is one of the highest in the world. The country fall into
another civil war that broke out in December, 2013. After the latest
round of the civil war, around 1.74 million people have been internally
displaced, and 640,000 people took refuges in the neighboring
countries\footnote{\url{https://reliefweb.int/report/south-sudan/south-sudan-humanitarian-bulletin-issue-6-16-july-2018}}
. The government had little control on the market, and the economy
experienced unprecedented hyperinflation over the last 5
years\ldots{}\ldots{}\ldots{}\ldots{}

\paragraph*{Works Cited}\label{works-cited}
\addcontentsline{toc}{paragraph}{Works Cited}

\hypertarget{refs}{}
\hypertarget{ref-Ahmed2009}{}
Ahmed, Akhter U, Mehnaz Rabbani, Munshi Sulaiman, and Narayan C. Das.
2009. \emph{The Impact of Asset Transfer on Livelihoods of the Ultra
Poor in Bangladesh}. Vol. 7. 39.

\hypertarget{ref-Ahmed2007}{}
Ahmed, Akhter U., Ruth Vargas Hill, Lisa C. Smith, Doris M. Wiesmann,
and Tim Frankenberger. 2007. ``The World's Most Deprived;
Characteristics and Causes of Extreme Poverty and Hunger.'' October.

\hypertarget{ref-Bandiera2017}{}
Bandiera, Oriana, Robin Burgess, Narayan Das, Selim Gulesci, Imran
Rasul, and Munshi Sulaiman. 2017. ``Labor markets and poverty in village
economies.'' \emph{Quarterly Journal of Economics} 132 (2): 811--70.
doi:\href{https://doi.org/10.1093/qje/qjx003}{10.1093/qje/qjx003}.

\hypertarget{ref-Bandiera2013}{}
Bandiera, Oriana, Robin Burgess, Narayan Das, Selim Gulesci, Imran
Rasul, Munshi Sulaiman, Francisco Buera, et al. 2013. ``Can basic
entrepreneurship transform the economic lives of the poor?''
\emph{Working Paper}, no. April.

\hypertarget{ref-Voropaeva1966}{}
Banerjee, Abhijit, Esther Duflo, Raghabendra Chattopadhyay, and Jeremy
Shapiro. 2011. ``Targeting the Hard-Core Poor: An Impact Assessment.''

\hypertarget{ref-Banerjee2015}{}
Banerjee, Abhijit, Esther Duflo, Nathanael Goldberg, Dean Karlan, Robert
Osei, William Parienté, Jeremy Shapiro, Bram Thuysbaert, and Christopher
Udry. 2015. ``A multifaceted program causes lasting progress for the
very poor: Evidence from six countries.'' \emph{Science} 348 (6236).
doi:\href{https://doi.org/10.1126/science.1260799}{10.1126/science.1260799}.

\hypertarget{ref-Blattman2016}{}
Blattman, Christopher, Eric P. Green, Julian Jamison, M. Christian
Lehmann, and Jeannie Annan. 2016. ``Blattman 2015.'' \emph{American
Economic Journal: Applied Economics} 8 (2): 35--64.
doi:\href{https://doi.org/10.1257/app.20150023}{10.1257/app.20150023}.

\hypertarget{ref-Dreze1990}{}
Dreze, J. 1990. ``Poverty in India and the IRDP Delusion.''
\emph{Economic and Political Weekly} 25 (39): A95--A104.
\href{http://www.jstor.org/discover/10.2307/4396805?uid=3738456\%7B/\&\%7Duid=2\%7B/\&\%7Duid=4\%7B/\&\%7Dsid=21102145880963}{http://www.jstor.org/discover/10.2307/4396805?uid=3738456\{\textbackslash{}\&\}uid=2\{\textbackslash{}\&\}uid=4\{\textbackslash{}\&\}sid=21102145880963}.

\hypertarget{ref-Evans1999}{}
Evans, Timothy G., Alayne M. Adams, Rafi Mohammed, and Alison H. Norris.
1999. ``Demystifying nonparticipation in microcredit: A population-based
analysis.'' \emph{World Development} 27 (2): 419--30.
doi:\href{https://doi.org/10.1016/S0305-750X(98)00134-X}{10.1016/S0305-750X(98)00134-X}.

\hypertarget{ref-Hashemi2011}{}
Hashemi, Syed M, and Wamiq Umaira. 2011. ``New pathways for the
poorest\_Hashemi,S.M. \& Umaira, W. (2011),'' no. January: 1--20.

\hypertarget{ref-Kaur2016}{}
Kaur, Prabhjot. 2016. ``Efficiency of microfinance institutions in
India: Are they reaching the poorest of the poor?'' \emph{Vision} 20
(1): 54--65.
doi:\href{https://doi.org/10.1177/0972262916628988}{10.1177/0972262916628988}.

\hypertarget{ref-Mallick2013}{}
Mallick, Debdulal. 2013. ``How Effective is a Big Push to the Small?
Evidence from a Quasi-Experiment.'' \emph{World Development} 41 (1).
Elsevier Ltd: 168--82.
doi:\href{https://doi.org/10.1016/j.worlddev.2012.05.021}{10.1016/j.worlddev.2012.05.021}.

\hypertarget{ref-Matin2003}{}
Matin, Imran, and David Hulme. 2003. ``Programs for the Poorest:
Learning from the IGVGD Program in Bangladesh.'' \emph{World
Development} 31 (3): 647--65.
doi:\href{https://doi.org/10.1016/S0305-750X(02)00223-1}{10.1016/S0305-750X(02)00223-1}.

\hypertarget{ref-Rabbani2006}{}
Rabbani, Mehnaz, Munshi Sulaiman, and Vivek A. Prakash. 2006. ``Impact
Assessment of CFPR/TUP: A Descriptive Analysis Based on 2002-2005 Panel
Data.'' \emph{CFPR/TUP Working Paper} 12 (July): 1--35.
\href{https://www.researchgate.net/profile/Munshi\%7B/_\%7DSulaiman/publication/46476816\%7B/_\%7DImpact\%7B/_\%7DAssessment\%7B/_\%7Dof\%7B/_\%7DCFPRTUP\%7B/_\%7DA\%7B/_\%7DDescriptive\%7B/_\%7DAnalysis\%7B/_\%7DBased\%7B/_\%7Don\%7B/_\%7D2002-2005\%7B/_\%7DPanel\%7B/_\%7DData/links/5750604e08ae1f765f9265e4/Impact-Assessment-of-CFPR-TUP-A-Descriptive-Analysis-Based-on-2002-2}{https://www.researchgate.net/profile/Munshi\{\textbackslash{}\_\}Sulaiman/publication/46476816\{\textbackslash{}\_\}Impact\{\textbackslash{}\_\}Assessment\{\textbackslash{}\_\}of\{\textbackslash{}\_\}CFPRTUP\{\textbackslash{}\_\}A\{\textbackslash{}\_\}Descriptive\{\textbackslash{}\_\}Analysis\{\textbackslash{}\_\}Based\{\textbackslash{}\_\}on\{\textbackslash{}\_\}2002-2005\{\textbackslash{}\_\}Panel\{\textbackslash{}\_\}Data/links/5750604e08ae1f765f9265e4/Impact-Assessment-of-CFPR-TUP-A-Descriptive-Analysis-Based-on-2002-2}.

\hypertarget{ref-Thorp2005}{}
Thorp, Rosemary, Frances Stewart, and Amrik Heyer. 2005. ``When and how
far is group formation a route out of chronic poverty?'' \emph{World
Development} 33 (6): 907--20.
doi:\href{https://doi.org/10.1016/j.worlddev.2004.09.016}{10.1016/j.worlddev.2004.09.016}.

\hypertarget{ref-WHO2014}{}
WHO. 2014. ``Republic of South Sudan Fact Sheet: Emergency Risk and
Crisis Management,'' no. JANUARY: 2013--5.
\href{http://www.who.int/hac/crises/ssd/sitreps/south\%7B/_\%7Dsudan\%7B/_\%7Dcountry\%7B/_\%7Dfact\%7B/_\%7Dsheet\%7B/_\%7DMarch2014.pdf}{http://www.who.int/hac/crises/ssd/sitreps/south\{\textbackslash{}\_\}sudan\{\textbackslash{}\_\}country\{\textbackslash{}\_\}fact\{\textbackslash{}\_\}sheet\{\textbackslash{}\_\}March2014.pdf}.

\hypertarget{ref-WorldBank2018}{}
World Bank. 2018. ``Poverty and Shared Prosperity 2018,'' 19--48.
doi:\href{https://doi.org/10.1596/978-1-4648-0958-3}{10.1596/978-1-4648-0958-3}.


\end{document}
